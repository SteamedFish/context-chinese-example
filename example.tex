% -*- mode: ConTeXt; -*-
% vim: filetype=context

% 设置环境语言为中文,也就是排版规则使用中文的各种规则。
\mainlanguage[cn]
\language[cn]
\enableregime[utf]
\setscript[hanzi]

% 开启颜色支持
\usecolors[xwi]
\setupcolors[state=start]

% 设置图片的默认路径
\setupexternalfigures[directory={template, /Users/steamedfish/dumbo/dumbo-training/template}]

% 代码语法高亮方案,需要系统中拥有 vim 包并且 vim 在 $PATH 中
\usemodule[t-vim]
\definevimtyping [PYTHON] [syntax=python, numbering=yes, numberlocation=right, numbercolor=darkyellow, margin=3em, lines=split,]
\definevimtyping [BASH] [syntax=sh, numbering=yes, numberlocation=right, numbercolor=darkyellow, margin=3em, lines=split,]

% 流程图模块,后面会使用到
\usemodule[chart]

% 设置 pdf 的超链接样式,以及侧边栏超链接(menu)
\setupinteraction [
  state=start,
  style=bold,
  color=darkblue,
  click=yes,
  menu=on,
  title={\CONTEXT 测试文档},
  subtitle={subtitle},
  author={牛逼的作者},
]

% 设置页面参数
\setuppapersize[A4]

% 设置列表样式
\setupitemize[textdistance=medium,]
\setupitemize[autointro]    % prevent orphan list intro
\setupitemize[indentnext=no]

% 设置目录和书签,并且将书签目录默认设置为显示状态,并且默认只显示最顶上两个级别
\placebookmarks[chapter, section, subsection, subsubsection, subsubsubsection, subsubsubsubsection][chapter, section]
\setupinteractionscreen[option=bookmark]
\setuptagging[state=start]

% 设置文本字体的样式
\definefontfeature[default][default][script=latn, protrusion=quality, expansion=quality, itlc=yes, textitalics=yes, onum=yes, pnum=yes]
\definefontfeature[smallcaps][script=latn, protrusion=quality, expansion=quality, smcp=yes, onum=yes, pnum=yes]
\setupalign[hz,hanging]
\setupitaliccorrection[global, always]
\setupbodyfontenvironment[default][em=italic]
% \definefallbackfamily[mainface][rm][DejaVu Serif][preset=range:greek, force=yes]
\definefontfamily[mainface][rm][notosanscjksc][]
\definefontfamily[mainface][mm][notosanscjksc]
\definefontfamily[mainface][ss][notosanscjksc]
\definefontfamily[mainface][tt][notosansmonocjksc][features=none]
\setupbodyfont[mainface,]

% 设置文本空白
\setupwhitespace[medium]

% 设置行间距
\setupinterlinespace[medium]

% 设置数字使用中文而不是阿拉伯文字
\definestructureconversionset[chinese][numbers][cn]

% 设置章节标题的样式,我们让章的编号用中文,而小节编号用阿拉伯数字。
% 章节必须从奇数页开始,并且章节后面会有一个横线
% 章节的页码上,设置不要有 header
\define[2]\ChineseChapter{\color[darkslateblue]{第#1章} #2}
\define[2]\ChineseSection{\color[cornflowerblue]{第 #1 节} #2}
\setuphead[title]              [style=\tfd, header=empty]
\setuphead[chapter]            [numbercommand=\ChineseChapter, style=\bfd, page=makeup, textcolor=darkslateblue, align=middle, after=\hairline, before=, sectionconversionset=chinese, header=empty]
\setuphead[section]            [style=\bfb, numbercommand=\ChineseSection, textcolor=cornflowerblue, align=normal, after=, before=]
\setuphead[subsection]         [style=\bf, numbercommand=\ChineseSection, textcolor=cornflowerblue, align=normal, after=, before=]
\setuphead[subsubsection]      [style=\bf, textcolor=cornflowerblue]
\setuphead[subsubsubsection]   [style=\sc]
\setuphead[subsubsubsubsection][style=\it]

% 设置目录列表的样式
% 目录页面,设置为没有 header
\setupcombinedlist[content][list={chapter,section}, alternative=c, header=no]
\define[1]\ChineseContentPage{第#1页}
\define[2]\ChineseContentChapterNumber{第#1章 #2}
\define[1]\ChineseContentChapter{#1}
\setuplist[chapter][width=2.15cm, pagestyle={\feature[+][tabularnumbers]}, style=bold, headnumber=yes, sectionnumber=no, numbercommand=\ChineseContentChapterNumber, pagecommand=\ChineseContentPage, aligntitle=yes, textcommand=\ChineseContentChapter]
\setuplist[section][pagestyle={\feature[+][tabularnumbers]}, style=regular, headnumber=yes, sectionnumber=yes, pagecommand=\ChineseContentPage, aligntitle=yes, margin=1em]
\setuplist[subsection][width=15mm, headnumber=no, sectionnumber=no, pagecommand=\ChineseContentPage, margin=14pt]
\setuplist[subsubsection][width=20mm, style=slanted, pagestyle=normal]


\definedescription
  [description]
  [headstyle=bold, style=normal, location=hanging, width=broad, margin=1cm, alternative=hanging]

% 设置图片和表格的样式
\setupfloat[figure][default={here, nonumber}]
\setupfloat[table][default={here, nonumber}]

% 设置分割线的样式
\setupthinrules[interlinespace=small, color=darkslateblue, height=3pt] % width of horizontal rules

% 开始正文内容
\starttext

% 标题页面部分,应该关闭页码显示,关闭章节标题,并且更改字体设置
\setuppagenumber[state=stop]
\setuphead[chapter]            [text=none, numbercommand=, before=, after=\hairline, sectionnumber=no, ]
\switchtobodyfont[20pt]

% 标题页面
\startalignment[middle]
  {\tfd \CONTEXT 示例文档}
  \smallskip
  {\tfa 副标题}
  \smallskip
  {\tfa 牛逼的作者}
  \bigskip
\stopalignment

% 目录部分,应该重置页面大小,并且把页码设置为罗马数字
\switchtobodyfont[14pt]
\page[yes, blank, odd]
\setupbackgrounds[page][background=none, state=start]
\setuppagenumber [state=start, way=bychapter, numberconversion=romannumerals]
% 设置目录部分的页脚显示罗马数字
\setupfootertexts
[{\color[darkgray]{\userpagenumber}}]

% 目录、表格列表、图片列表(后两者本文没有,注释掉)
\completecontent
% \completelistoftables
% \completelistoffigures

% 每个章节应该从奇数页开始
\page[yes, blank, odd]

% 正文部分,重新打开章节标题显示,并且把页码设置为中文数字,重新从 1 开始计数
\setuphead[chapter]            [numbercommand=\ChineseChapter, style=\bfd, page=makeup, textcolor=darkslateblue, align=middle, after=\hairline, before=, sectionconversionset=chinese, header=empty]
\setuppagenumbering[conversion=chinese,location=footer]
\setuppagenumber [numberconversion=cn,state=start, way=bytext]
\setcounter[userpage][1]

% 设置页面的 header
% 为了照顾双面打印,header 的奇数页和偶数页设置应该不同,一个为章节内容,一个为书名。
\setupheadertexts
[]
[{\color[darkgray]{第\getmarking[chapternumber]章 --- \getmarking[chapter]}}]
[]
[{\color[darkgray]{\CONTEXT 示例文档}}]

% 设置页面的 footer,所有页面均相同,居中显示。
\setupfootertexts
[{\color[darkgray]{第\userpagenumber 页}}]

% 正式开始正文

\chapter[title={概述},reference={概述}]

本文为 \CONTEXT 的中文演示文档。

\section[title={前言},reference={前言}]

牛逼的前言。

\chapter[title={牛逼},reference={牛逼}]

牛逼的一章。

\section[title={厉害},reference={厉害}]

厉害了我的哥

\subsection[title={语法高亮},reference={语法高亮}]

代码块:

\startBASH
  if [ "$OSTYPE" == "linux-gnu" ]; then
      echo "your system is niubility"
  elif [[ "$OSTYPE" == "darwin"* ]]; then
      echo "your system is zhuangbility"
  else
      echo "your system is shability"
  fi
\stopBASH

在线高亮代码:\inlinePYTHON{import sys; print(next(p for p in sys.path))},牛逼吧!

\subsection[title={列表},reference={列表}]

\startitemize[n]
\item
  列表一
  \startitemize[a,packed]
  \item
    aaa
  \item
    bbb
  \item
    ccc
  \stopitemize
\item
  列表二
  \startitemize[a,packed]
  \item
    dfafasd
  \item
    我的神呐
  \stopitemize
\stopitemize

\section[title={流程图},reference={流程图}]

% 文档参见 http://www.pragma-ade.com/general/manuals/charts-mkiv

\startFLOWchart [666]
  \startFLOWcell
    \name {1}
    \location {1,1}
    \shape {34}
    \text {\tt{\CONTEXT 牛逼吧!}}
    \connection [rl] {2}
    \connection [bt] {3}
  \stopFLOWcell
  \startFLOWcell
    \name {3}
    \location {1,2}
    \shape {34}
    \text {确实牛逼啊!}
    \connection [rl] {2}
  \stopFLOWcell
  \startFLOWcell
    \name {2}
    \location {2,1}
    \shape {34}
    \text {\tt{666} \\ \tt{实在是 666 啊!}}
  \stopFLOWcell
\stopFLOWchart

\FLOWchart [666]


\section[title={表格},reference={表格}]


\starttabulate[|c|r|r|]
\HL
\NC 0 \NC \setupunit[method=0]\unit{00,000.10 kilogram}
      \NC \setupunit[method=0]\unit{@@,@@0.10 kilogram} \NC \NR
\NC 1 \NC \setupunit[method=1]\unit{00,000.10 kilogram}
      \NC \setupunit[method=1]\unit{@@,@@0.10 kilogram} \NC \NR
\NC 2 \NC \setupunit[method=2]\unit{00,000.10 kilogram}
      \NC \setupunit[method=2]\unit{@@,@@0.10 kilogram} \NC \NR
\NC 3 \NC \setupunit[method=3]\unit{00,000.10 kilogram}
      \NC \setupunit[method=3]\unit{@@,@@0.10 kilogram} \NC \NR
\NC 4 \NC \setupunit[method=4]\unit{00,000.10 kilogram}
      \NC \setupunit[method=4]\unit{@@,@@0.10 kilogram} \NC \NR
\NC 5 \NC \setupunit[method=5]\unit{00,000.10 kilogram}
      \NC \setupunit[method=5]\unit{@@,@@0.10 kilogram} \NC \NR
\NC 6 \NC \setupunit[method=6]\unit{00,000.10 kilogram}
      \NC \setupunit[method=6]\unit{@@,@@0.10 kilogram} \NC \NR
\HL
\stoptabulate

\section[title={文字},reference={文字}]

各种角度 \rotate[rotation=90]{right} 的翻转  \rotate[rotation=270]{哎呀呀}

\setupframedtexts
      [width=.8\makeupwidth,
       background=color,
       backgroundcolor=gray,
       corner=round,
       framecolor=blue,
       rulethickness=2pt]
  \startframedtext
    这真的是太牛逼了
    \blank
    不知道应该用什么词汇来形容
\stopframedtext

\definecolor [darkorange]   [c=0.0,m=0.60,y=1.00,k=0.0]

\blackrule[width=\hsize,height=1cm,color=red,after=]
\blackrule[width=\hsize,height=1cm,color=white,after=]
\blackrule[width=\hsize,height=1cm,color=blue,after=]
\blackrule[width=\hsize,height=1cm,color=darkorange]

\section[title={pictures},reference={pictures}]

\startMPrun
     input example ;
\stopMPrun

\externalfigure[mprun.21][width=10cm]

\stoptext
